\documentclass{article}

\begin{document}


\section{Usage}
The usage of the compiler is quite simple, to compile it go to the root of the project and execute \texttt{make -C src/}, then you just need to use the given script \texttt{./jlc <sourcefile>}.

\section{Language specification}
The language is specified using \texttt{BNFC}, just open the \texttt{Javalette.cf} file in order to check all the grammar rules of the language.

\section{Shift/reduce conflicts}

\section{Implemented extensions}

We have implemented 5 extensions:

\begin{itemize}
  \item {\bf Arrays}
  \item {\bf Multidimmensional arrays}
  \item {\bf Structures}
  \item {\bf Classes}
  \item {\bf Classes 2 (with dynamic dispatch)}
\end{itemize}

\section{About the compiler}

The compiler consists of the following pipeline:

\begin{enumerate}
 \item {\bf Lexer}: Generated with \texttt{BNFC}, this phase reads the source file and translate it into a set of tokens.
 \item {\bf Desugarer}: Removes syntax-sugar such as for loops and structure/classes definitions.
 \item {\bf Typechecker}: Checks the program is correct, and returns the typed program.
 \item {\bf Code generator}: Using the typed program, generated the \texttt{LLVM} code necessary to run the program.
\end{enumerate}

\end{document}
